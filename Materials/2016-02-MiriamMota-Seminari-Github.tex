%%%%%%%%%%%%%%%%%%%%%%%%%%%%%%%%%%%%%%%%%
% Beamer Presentation
% LaTeX Template
% Version 1.0 (10/11/12)
%
% This template has been downloaded from:
% http://www.LaTeXTemplates.com
%
% License:
% CC BY-NC-SA 3.0 (http://creativecommons.org/licenses/by-nc-sa/3.0/)
%
%%%%%%%%%%%%%%%%%%%%%%%%%%%%%%%%%%%%%%%%%

%----------------------------------------------------------------------------------------
%	PACKAGES AND THEMES
%----------------------------------------------------------------------------------------

\documentclass{beamer}

\mode<presentation> {

% The Beamer class comes with a number of default slide themes
% which change the colors and layouts of slides. Below this is a list
% of all the themes, uncomment each in turn to see what they look like.

%\usetheme{default}
%\usetheme{AnnArbor}
%\usetheme{Antibes}
%\usetheme{Bergen}
%\usetheme{Berkeley}
%\usetheme{Berlin}
%\usetheme{Boadilla}
%\usetheme{CambridgeUS}
%\usetheme{Copenhagen}
%\usetheme{Darmstadt}
%\usetheme{Dresden}
%\usetheme{Frankfurt}
%\usetheme{Goettingen}
%\usetheme{Hannover}
%\usetheme{Ilmenau}
%\usetheme{JuanLesPins}
%\usetheme{Luebeck}
%\usetheme{Madrid}
%\usetheme{Malmoe}
%\usetheme{Marburg}
%\usetheme{Montpellier}
%\usetheme{PaloAlto}
%\usetheme{Pittsburgh}
%\usetheme{Rochester}
%\usetheme{Singapore}
%\usetheme{Szeged}
\usetheme{Warsaw}

% As well as themes, the Beamer class has a number of color themes
% for any slide theme. Uncomment each of these in turn to see how it
% changes the colors of your current slide theme.

%\usecolortheme{albatross}
%\usecolortheme{beaver}
%\usecolortheme{beetle}
%\usecolortheme{crane}
%\usecolortheme{dolphin}
%\usecolortheme{dove}
%\usecolortheme{fly}
%\usecolortheme{lily}
%\usecolortheme{orchid}
%\usecolortheme{rose}
%\usecolortheme{seagull}
%\usecolortheme{seahorse}
%\usecolortheme{whale}
%\usecolortheme{wolverine}
\definecolor{magentavhir}{rgb}{0.9,0.1,0.8}
\usecolortheme[named=magentavhir]{structure}

%\setbeamertemplate{footline} % To remove the footer line in all slides uncomment this line
%\setbeamertemplate{footline}[page number] % To replace the footer line in all slides with a simple slide count uncomment this line

%\setbeamertemplate{navigation symbols}{} % To remove the navigation symbols from the bottom of all slides uncomment this line
}

\usepackage{ amssymb }
\usepackage{ textcomp }
\usepackage{graphicx} % Allows including images
\usepackage{booktabs} % Allows the use of \toprule, \midrule and \bottomrule in tables
\usepackage{amssymb}
\usepackage{textcomp}
\usepackage{marvosym}
\usepackage{hyperref}
\usepackage{color}
\newcommand{\shellcmd}[1]{{\color{blue}\indent\indent\texttt{\footnotesize\ #1}\\}}


%----------------------------------------------------------------------------------------
%	TITLE PAGE
%----------------------------------------------------------------------------------------

% The short title appears at the bottom of every slide, the full title is only on the title page
\title[Git-Github]{Git - Github}

\author[Miriam Mota ]{\includegraphics[height=1cm,width=4cm]{images/UEBblanc.jpg} \\ Miriam Mota} % Your name
\institute%[UEB - VHIR] % Your institution as it will appear on the bottom of every slide, may be shorthand to save space
{
Statistics and Bioinformatics Unit (UEB),\\ 
Vall d'Hebron Research Institute (VHIR) \\ % Your institution for the title page
\medskip
 % Your email address
\textit{miriam.mota@vhir.org}
}
\date{\today} % Date, can be changed to a custom date

\begin{document}

\begin{frame}
\titlepage % Print the title page as the first slide
\end{frame}

\begin{frame}
\frametitle{Resumen} % Table of contents slide, comment this block out to remove it
\small{\tableofcontents} % Throughout your presentation, if you choose to use \section{} and \subsection{} commands, these will automatically be printed on this slide as an overview of your presentation
\end{frame}

%----------------------------------------------------------------------------------------
%	PRESENTATION SLIDES
%----------------------------------------------------------------------------------------

%------------------------------------------------
\section{Introducci\'o}


%------------------------------------------------
\begin{frame}
	\frametitle{Git-Github}
	\begin{itemize}
		\item \textbf{Git} : Eina de control de versions.
		\item \textbf{GitHub.com} : p\`{a}gina web que es conecta als repositoris per a tenir-los en l\'inea i interactuar amb ells.
		\item \textbf{GitHub Desktop} : aplicaci\'o que es pot instal.lar en Windows per ajudar a sincronitzar el codi local amb GitHub.com.
	\end{itemize}
\end{frame}
%------------------------------------------------

%------------------------------------------------
\begin{frame}
	\frametitle{Definicions}
	\begin{itemize}
		\item \textbf{Repositori $\Longleftrightarrow$ Projecte}. On tindrem...
		\begin{itemize}
			\item Carpetes, arxius, dades, etc.
			\item README: Descripció del projecte
		\end{itemize}
		\item \textbf{Branques}: Ens perment treballar en diferents versions d'un repositori alhora
		\item \textbf{Organitzacions de treball}
	\end{itemize}
\end{frame}
%------------------------------------------------


\section{Instal.laci\'o}
\subsection{Compte Github} 
%------------------------------------------------
\begin{frame}
\frametitle{Creem nou compte}
\url{https://github.com} \hspace{1cm} 
	\begin{figure}
		\includegraphics[width=0.62\linewidth]{images/Selection_187.png}
	\end{figure}
\end{frame}
%------------------------------------------------


%------------------------------------------------
\begin{frame}
	\frametitle{Compte UEB}
	\url{https://github.com/uebvhir}
	\begin{figure}
		\includegraphics[width=0.77\linewidth]{images/Selection_188.png}
	\end{figure}
\end{frame}
%------------------------------------------------



%------------------------------------------------
\begin{frame}
	\frametitle{Creem repositori}
	\begin{figure}
		\includegraphics[width=1\linewidth]{images/Selection_189.png}
	\end{figure}
\end{frame}
%------------------------------------------------




%------------------------------------------------

\subsection{Intalaci\'o Git}
\begin{frame}
	
		\begin{block}{Linux}
			\begin{itemize}
				\item \textbf{Software necessari:} \\
				\shellcmd{sudo apt-get update}
				\shellcmd{sudo apt-get install git} \vspace{0.2cm}
				\item \textbf{Software opcional:} \\
				\shellcmd{sudo add-apt-repository ppa:rabbitvcs/ppa}
				\shellcmd{sudo apt-get install rabbitvcs-cli rabbitvcs-core rabbitvcs-gedit rabbitvcs-nautilus3}
			\end{itemize}
			
		\end{block}
		
		

		\begin{block}{Windows}
			\begin{itemize}
				\item \textbf{Necessari}: Instalem Git (\url{https://git-for-windows.github.io/})
				\item \textbf{Opcional:}  Instalem Github \url {https://desktop.github.com/})
			\end{itemize}
			{\small Obrim Git CMD}
		\end{block}

	
\end{frame}


\begin{frame}
	\frametitle{Configuraci\'o Git}
	
	\begin{itemize}
		\item Configurar nom i correu\\
		\shellcmd{git config --global user.name ``Nom''}
		\shellcmd{git config --global user.email ``Correu''}
		\item Comprovem dades\\ 
		\shellcmd{git config --list}
		
	\end{itemize}
	
	

\end{frame}


\section{Exemple}
%------------------------------------------------
\subsection{Repositori Github - Local}
\begin{frame}
	\frametitle{Clonem repositori a l'ordinador}
	\begin{itemize}
		\item Ens posem al directori on volem que es cloni el repositori
		\item Clonem\\
		\shellcmd{git clone https://github.com/uebvhir/exempleSeminari.git}
		\shellcmd{cd exempleSeminari}\vspace{0.2cm}
		\item Podem afegir arxius des de: 
		\begin{enumerate}
			\item Local
			\item Github
		\end{enumerate}
	\end{itemize}	
\end{frame}

\subsection{Afegir arxius}
%------------------------------------------------
%\subsection{Afegir arxius en local}
\begin{frame}
	\frametitle{Nous arxius local}
	\begin{itemize}
		\item Creem o copiem arxius a la carpeta exempleSeminari
		\item Actualitzem canvis fets en local a github\\
		\shellcmd{git add . (afegeix tots els arxius nous)\\
			\hspace{1.5cm}-u (fitxers que han canviat de nom o eliminats)\\
			\hspace{1.5cm}-A (fa totes dues coses)}
		\shellcmd{git commit -m "missatge de canvis"}
		\shellcmd{git push -u origin master}

	\end{itemize}	
\end{frame}


%------------------------------------------------
%\subsection{Afegir arxius al github}
\begin{frame}
	\frametitle{Nous arxius github}
	\begin{itemize}
		\item Creem o copiem arxius a la carpeta exempleSeminari des de github
		\item Actualitzem canvis fets a github en local\\
		\shellcmd{git add -A }
		\shellcmd{git commit -m "missatge de canvis"}
		\shellcmd{git pull --rebase origin master}
		
	\end{itemize}	
\end{frame}


%------------------------------------------------
%\subsection{Nou repositori local}
\begin{frame}
	\frametitle{Repositori local - github}
	\begin{itemize}
		\item Creem repositori local \\
		\shellcmd{mkdir nomRepo}
		\shellcmd{cd nomRepo}
		\shellcmd{git init}
		\shellcmd{git remote add origin https://github.com/UsuariGithub/nomRepo.git}
		\item Afegim arxius
		\item Creem repositori amb el mateix nom que el local
		\item Sincronitzem
		
		\shellcmd{git add -A}
		\shellcmd{git commit -m "nou repositori afegit"}
		\shellcmd{git push -u origin master}
		
		
	\end{itemize}	
\end{frame}


%------------------------------------------------
\subsection{Branques}
\begin{frame}
	\frametitle{Branques}
	\begin{itemize}
		\item Afegir branca \shellcmd{ git checkout -b brancaMiriam}
		\item ELIMINAR branca  \shellcmd{ git branch -d brancaMiriam}
		\item Veure en quina branca estem  \shellcmd{ git branch}
		\item Actualitzar branca al github \shellcmd{ git push origin brancaMiriam}
		\item Canviar de branca \shellcmd{git checkout master}
	\end{itemize}
		\begin{figure}
			\includegraphics[width=0.6\linewidth]{images/Selection_194.png}
		\end{figure}
\end{frame}

%------------------------------------------------
\subsection{Organitzaci\'o}
\begin{frame}
	\frametitle{Organitzacions}
	\begin{columns}[c] % The "c" option specifies centered vertical alignment while the "t" option is used for top vertical alignment
		
		\column{.15\textwidth} % Left column and width
			\begin{figure}
				\includegraphics[width=1\linewidth]{images/Selection_192.png}
			\end{figure}
		
		
		\column{.85\textwidth} % Right column and width
			\begin{figure}
				\includegraphics[width=1.2\linewidth]{images/Selection_193.png}
			\end{figure}
	
	\end{columns}
\end{frame}



\begin{frame}
	\frametitle{Altres comandes}
	\begin{itemize}
		\item Colors especials per a la consola \shellcmd{git config color.ui true}
		\item Agregar arxius de forma interactiva  \shellcmd{git add -i}
		\item Canvis des de l'\'ultim commit \shellcmd{git status}
		\item Fixar proxy \shellcmd{git config --global http.proxy http://conf\_www.ir.vhebron.net:8081}
		\item Desfer un git init \shellcmd{rm -rf .git}
		\item Veure canvis en els diferents commits \shellcmd{git diff origin/master}
	\end{itemize}
\end{frame}

\section{Bibliograf\'ia}
\begin{frame}
	\begin{itemize}
		\item \textbf{Rstudio - Git - Github:} \url{http://www.r-bloggers.com/rstudio-and-github/}
		\item \textbf{Tutorial online: } \url{https://try.github.io/levels/1/challenges/1}
		\item \textbf{Pro Git:} \url{https://git-scm.com/book/es/v1}
	\end{itemize}
\end{frame}






%------------------------------------------------
%\section{Bibliograf\'ia}
%------------------------------------------------

%\subsection{Bibliography}

%------------------------------------------------

%\begin{frame}
%\frametitle{Referencias}

%\tiny{
%\begin{thebibliography}{99} % Beamer does not support BibTeX so references must be inserted manually as below

%\bibitem M Slawski, M Daumer and A-L Boulesteix (2008) 
%CMA – a comprehensive Bioconductor package for supervised classification with high dimensional data {\it BMC Bioinformatics} 9:439. 


%\end{thebibliography}
%}

%\end{frame}


%------------------------------------------------


%------------------------------------------------



%------------------------------------------------

%\begin{frame}
%\titlepage % Print the title page also as the last slide
%\end{frame}

%----------------------------------------------------------------------------------------

\end{document} 